\documentclass[10pt]{article}

\usepackage{graphics}
\usepackage{hevea}
\usepackage{verbatim}


\newcommand{\logoscale}{0.7}
\newcommand{\imgscale}{0.58}
\newcommand{\insimg}[1]{\insscaleimg{\imgscale}{#1}}

\newcommand{\insscaleimg}[2]{
  \imgsrc{#2}{}
  \begin{latexonly}
    \scalebox{#1}{\includegraphics{#2}}
  \end{latexonly}
}

\newcommand{\ns}[1]{\texttt{#1}}
\newcommand{\ejabberd}{\texttt{ejabberd}}
\newcommand{\Jabber}{Jabber}

\newcommand{\modregister}{\texttt{mod\_register}}
\newcommand{\modroster}{\texttt{mod\_roster}}
\newcommand{\modconfigure}{\texttt{mod\_configure}}
\newcommand{\moddisco}{\texttt{mod\_disco}}
\newcommand{\modstats}{\texttt{mod\_stats}}
\newcommand{\modvcard}{\texttt{mod\_vcard}}
\newcommand{\modoffline}{\texttt{mod\_offline}}
\newcommand{\modecho}{\texttt{mod\_echo}}
\newcommand{\modprivate}{\texttt{mod\_private}}
\newcommand{\modtime}{\texttt{mod\_time}}
\newcommand{\modversion}{\texttt{mod\_version}}

\newcommand{\tjepref}[2]{\footahref{http://www.jabber.org/jeps/jep-#1.html}{#2}}
\newcommand{\jepref}[1]{\tjepref{#1}{JEP-#1}}
%\setcounter{tocdepth}{3}


\title{Ejabberd Developers Guide}
\author{Alexey Shchepin \\
  \ahrefurl{mailto:alexey@sevcom.net} \\
  \ahrefurl{xmpp:aleksey@jabber.ru}}
\date{September 10, 2003}

\begin{document}
\begin{titlepage}
  \maketitle{}
  
  {\centering
    \insscaleimg{\logoscale}{logo.png}
    \par
  }
\end{titlepage}
%\newpage
\tableofcontents{}

\newpage
\section{Introduction}
\label{sec:intro}

\ejabberd{} is a Free and Open Source fault-tolerant distributed \Jabber{}
server.  It is written mostly in Erlang.

The main features of \ejabberd{} are:
\begin{itemize}
\item Works on most of popular platforms: *nix (tested on Linux, FreeBSD and
  NetBSD) and Win32
\item Distributed: You can run \ejabberd{} on a cluster of machines to let all of
  them serve one Jabber domain.
\item Fault-tolerance: You can setup an \ejabberd{} cluster so that all the
  information required for a properly working service will be stored
  permanently on more than one node.  This means that if one of the nodes
  crashes, then the others will continue working without disruption.
  You can also add or replace nodes ``on the fly''.
\item Support for virtual hosting
\item Built-in \tjepref{0045}{Multi-User Chat} service
\item Built-in IRC transport
\item Built-in \tjepref{0060}{Publish-Subscribe} service
\item Built-in Jabber Users Directory service based on users vCards
\item Built-in web-based administration interface
\item Built-in \tjepref{0025}{HTTP Polling} service
\item SSL support
\item Support for LDAP authentication
\item Ability to interface with external components (JIT, MSN-t, Yahoo-t, etc.)
\item Migration from jabberd14 is possible
\item Mostly XMPP-compliant
\item Support for \tjepref{0030}{Service Discovery}.
\item Support for \tjepref{0039}{Statistics Gathering}.
\item Support for \ns{xml:lang}
\end{itemize}
\ejabberd{} is a Free and Open Source fault-tolerant distributed \Jabber{}
server.  It is written mostly in Erlang.



\subsection{How it works}
\label{sec:howitworks}


A \Jabber{} domain is served by one or more \ejabberd{} nodes.  These nodes can
be run on different machines that are connected via a network.  They all must
have the ability to connect to port 4369 of all another nodes, and must have
the same magic cookie (see Erlang/OTP documentation, in other words the file
\texttt{\~{}ejabberd/.erlang.cookie} must be the same on all nodes). This is
needed because all nodes exchange information about connected users, S2S
connections, registered services, etc\ldots



Each \ejabberd{} node have following modules:
\begin{itemize}
\item router;
\item local router.
\item session manager;
\item S2S manager;
\end{itemize}


\subsubsection{Router}

This module is the main router of \Jabber{} packets on each node.  It routes
them based on their destinations domains.  It has two tables: local and global
routes.  First, domain of packet destination searched in local table, and if it
found, then the packet is routed to appropriate process.  If no, then it
searches in global table, and is routed to the appropriate \ejabberd{} node or
process.  If it does not exists in either tables, then it sent to the S2S
manager.


\subsubsection{Local Router}

This module routes packets which have a destination domain equal to this server
name.  If destination JID has a non-empty user part, then it routed to the
session manager, else it is processed depending on it's content.


\subsubsection{Session Manager}

This module routes packets to local users.  It searches for what user resource
packet must be sended via presence table.  If this resource is connected to
this node, it is routed to C2S process, if it connected via another node, then
the packet is sent to session manager on that node.


\subsubsection{S2S Manager}

This module routes packets to other \Jabber{} servers.  First, it checks if an
open S2S connection from the domain of the packet source to the domain of
packet destination already exists. If it is open on another node, then it
routes the packet to S2S manager on that node, if it is open on this node, then
it is routed to the process that serves this connection, and if a connection
does not exist, then it is opened and registered.




\section{XML representation}
\label{sec:xmlrepr}

Each XML stanza represented as following tuple:
\begin{verbatim}
XMLElement = {xmlelement, Name, Attrs, [ElementOrCDATA]}
        Name = string()
        Attrs = [Attr]
        Attr = {Key, Val}
        Key = string()
        Val = string()
        ElementOrCDATA = XMLElement | CDATA
        CDATA = {xmlcdata, string()}
\end{verbatim}
E.\,g. this stanza:
\begin{verbatim}
<message to='test@conference.example.org' type='groupchat'>
  <body>test</body>
</message>
\end{verbatim}
represented as following structure:
\begin{verbatim}
{xmlelement, "message",
    [{"to", "test@conference.example.org"},
     {"type", "groupchat"}],
    [{xmlelement, "body",
         [],
         [{xmlcdata, "test"}]}]}}
\end{verbatim}



\section{Module \texttt{xml}}
\label{sec:xmlmod}

\begin{description}
\item[\verb|element_to_string(El) -> string()|]
\begin{verbatim}
El = XMLElement
\end{verbatim}
  Returns string representation of XML stanza \texttt{El}.

\item[\verb|crypt(S) -> string()|]
\begin{verbatim}
S = string()
\end{verbatim}
  Returns string which correspond to \texttt{S} with encoded XML special
  characters.

\item[\verb|remove_cdata(ECList) -> EList|]
\begin{verbatim}
ECList = [ElementOrCDATA]
EList = [XMLElement]
\end{verbatim}
  \texttt{EList} is a list of all non-CDATA elements of ECList.



\item[\verb|get_path_s(El, Path) -> Res|]
\begin{verbatim}
El = XMLElement
Path = [PathItem]
PathItem = PathElem | PathAttr | PathCDATA
PathElem = {elem, Name}
PathAttr = {attr, Name}
PathCDATA = cdata
Name = string()
Res = string() | XMLElement
\end{verbatim}
  If \texttt{Path} is empty, then returns \texttt{El}.  Else sequentially
  consider elements of \texttt{Path}.  Each element is one of:
  \begin{description}
  \item[\verb|{elem, Name}|] \texttt{Name} is name of subelement of
    \texttt{El}, if such element exists, then this element considered in
    following steps, else returns empty string.
  \item[\verb|{attr, Name}|] If \texttt{El} have attribute \texttt{Name}, then
    returns value of this attribute, else returns empty string.
  \item[\verb|cdata|] Returns CDATA of \texttt{El}.
  \end{description}

\item[TODO:]
\begin{verbatim}
         get_cdata/1, get_tag_cdata/1
         get_attr/2, get_attr_s/2
         get_tag_attr/2, get_tag_attr_s/2
         get_subtag/2
\end{verbatim}
\end{description}




\section{\ejabberd{} modules}
\label{sec:emods}


\subsection{\verb|gen_mod| behaviour}
\label{sec:genmod}

TBD

\subsection{Module \verb|gen_iq_handler|}
\label{sec:geniqhandl}

The module \verb|gen_iq_handler| allows to easily write handlers for IQ packets
of particular XML namespaces that addressed to server or to users bare JIDs.

In this module the following functions are defined:
\begin{description}
\item[\verb|add_iq_handler(Component, NS, Module, Function, Type)|]
\begin{verbatim}
Component = Module = Function = atom()
NS = string()
Type = no_queue | one_queue | parallel
\end{verbatim}
  Registers function \verb|Module:Function| as handler for IQ packets that
  contain child of namespace \verb|NS| in \verb|Component|.  Queueing
  discipline is \verb|Type|.  There are at least two components defined:
  \begin{description}
  \item[\verb|ejabberd_local|] Handles packets that addressed to server JID;
  \item[\verb|ejabberd_sm|] Handles packets that addressed to users bare JIDs.
  \end{description}
\item[\verb|remove_iq_handler(Component, NS)|]
\begin{verbatim}
Component = atom()
NS = string()
\end{verbatim}
  Removes IQ handler for namespace \verb|NS| from \verb|Component|.
\end{description}

Handler function must have the following type:
\begin{description}
\item[\verb|Module:Function(From, To, IQ)|]
\begin{verbatim}
From = To = jid()
\end{verbatim}
\end{description}



\begin{verbatim}
-module(mod_cputime).

-behaviour(gen_mod).

-export([start/1,
         stop/0,
         process_local_iq/3]).

-include("ejabberd.hrl").
-include("jlib.hrl").

-define(NS_CPUTIME, "ejabberd:cputime").

start(Opts) ->
    IQDisc = gen_mod:get_opt(iqdisc, Opts, one_queue),
    gen_iq_handler:add_iq_handler(ejabberd_local, ?NS_CPUTIME,
                                  ?MODULE, process_local_iq, IQDisc).

stop() ->
    gen_iq_handler:remove_iq_handler(ejabberd_local, ?NS_CPUTIME).

process_local_iq(From, To, {iq, ID, Type, XMLNS, SubEl}) ->
    case Type of
        set ->
            {iq, ID, error, XMLNS,
             [SubEl, ?ERR_NOT_ALLOWED]};
        get ->
            CPUTime = element(1, erlang:statistics(runtime))/1000,
            SCPUTime = lists:flatten(io_lib:format("~.3f", CPUTime)),
            {iq, ID, result, XMLNS,
             [{xmlelement, "query",
               [{"xmlns", ?NS_CPUTIME}],
               [{xmlelement, "cputime", [], [{xmlcdata, SCPUTime}]}]}]}
    end.
\end{verbatim}


\subsection{Services}
\label{sec:services}

TBD


TODO: use \verb|proc_lib|
\begin{verbatim}
-module(mod_echo).

-behaviour(gen_mod).

-export([start/1, init/1, stop/0]).

-include("ejabberd.hrl").
-include("jlib.hrl").

start(Opts) ->
    Host = gen_mod:get_opt(host, Opts, "echo." ++ ?MYNAME),
    register(ejabberd_mod_echo, spawn(?MODULE, init, [Host])).

init(Host) ->
    ejabberd_router:register_local_route(Host),
    loop(Host).

loop(Host) ->
    receive
        {route, From, To, Packet} ->
            ejabberd_router:route(To, From, Packet),
            loop(Host);
        stop ->
            ejabberd_router:unregister_local_route(Host),
            ok;
        _ ->
            loop(Host)
    end.

stop() ->
    ejabberd_mod_echo ! stop,
    ok.
\end{verbatim}



\end{document}
