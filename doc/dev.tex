\documentclass[a4paper,10pt]{article}

%% Packages
\usepackage{graphics}
\usepackage{hevea}
\usepackage{makeidx}
\usepackage{verbatim}

%% Index
\makeindex
% Remove the index anchors from the HTML version to save size and bandwith.
\newcommand{\ind}[1]{\begin{latexonly}\index{#1}\end{latexonly}}

%% Images
\newcommand{\logoscale}{0.7}
\newcommand{\imgscale}{0.58}
\newcommand{\insimg}[1]{\insscaleimg{\imgscale}{#1}}
\newcommand{\insscaleimg}[2]{
  \imgsrc{#2}{}
  \begin{latexonly}
    \scalebox{#1}{\includegraphics{#2}}
  \end{latexonly}
}

%% Various
\newcommand{\ns}[1]{\texttt{#1}}
\newcommand{\ejabberd}{\texttt{ejabberd}}
\newcommand{\Jabber}{Jabber}

%% Modules
\newcommand{\module}[1]{\texttt{#1}}
\newcommand{\modadhoc}{\module{mod\_adhoc}}
\newcommand{\modannounce}{\module{mod\_announce}}
\newcommand{\modconfigure}{\module{mod\_configure}}
\newcommand{\moddisco}{\module{mod\_disco}}
\newcommand{\modecho}{\module{mod\_echo}}
\newcommand{\modirc}{\module{mod\_irc}}
\newcommand{\modlast}{\module{mod\_last}}
\newcommand{\modlastodbc}{\module{mod\_last\_odbc}}
\newcommand{\modmuc}{\module{mod\_muc}}
\newcommand{\modmuclog}{\module{mod\_muc\_log}}
\newcommand{\modoffline}{\module{mod\_offline}}
\newcommand{\modofflineodbc}{\module{mod\_offline\_odbc}}
\newcommand{\modprivacy}{\module{mod\_privacy}}
\newcommand{\modprivate}{\module{mod\_private}}
\newcommand{\modpubsub}{\module{mod\_pubsub}}
\newcommand{\modregister}{\module{mod\_register}}
\newcommand{\modroster}{\module{mod\_roster}}
\newcommand{\modrosterodbc}{\module{mod\_roster\_odbc}}
\newcommand{\modservicelog}{\module{mod\_service\_log}}
\newcommand{\modsharedroster}{\module{mod\_shared\_roster}}
\newcommand{\modstats}{\module{mod\_stats}}
\newcommand{\modtime}{\module{mod\_time}}
\newcommand{\modvcard}{\module{mod\_vcard}}
\newcommand{\modvcardldap}{\module{mod\_vcard\_ldap}}
\newcommand{\modvcardodbc}{\module{mod\_vcard\_odbc}}
\newcommand{\modversion}{\module{mod\_version}}

%% Title page
% Define ejabberd version here.
\newcommand{\version}{1.1.0}

\title{Ejabberd \version\ Developers Guide}
\author{Alexey Shchepin \\
  \ahrefurl{mailto:alexey@sevcom.net} \\
  \ahrefurl{xmpp:aleksey@jabber.ru}}

%% Options
\newcommand{\marking}[1]{#1} % Marking disabled
\newcommand{\quoting}[2][yozhik]{} % Quotes disabled
\newcommand{\new}{\begin{latexonly}\marginpar{\textsc{new}}\end{latexonly}} % Highlight new features
\newcommand{\improved}{\begin{latexonly}\marginpar{\textsc{improved}}\end{latexonly}} % Highlight improved features
\newcommand{\moreinfo}[1]{} % Hide details

%% Footnotes
\newcommand{\tjepref}[2]{\footahref{http://www.jabber.org/jeps/jep-#1.html}{#2}}
\newcommand{\jepref}[1]{\tjepref{#1}{JEP-#1}}

\begin{document}

\label{sec:titlepage}
\begin{titlepage}
  \maketitle{}

  \begin{center}
  {\insscaleimg{\logoscale}{logo.png}
    \par
  }
  \end{center}

  \begin{quotation}\textit{I can thoroughly recommend ejabberd for ease of setup --
  Kevin Smith, Current maintainer of the Psi project}\end{quotation}

\end{titlepage}

\tableofcontents{}

% Input introduction.tex
\section{Introduction}
\label{sec:intr}

\quoting{I just tried out ejabberd and was impressed both by ejabberd itself and the language it is written in, Erlang. --
Joeri} 

\ejabberd{} is a free (GPL) distributed fault-tolerant \Jabber{}/XMPP server and is mainly written in \footahref{http://www.erlang.org/}{Erlang}.

\ejabberd{} is designed to be a \marking{stable}, \marking{standards compliant}, and \marking{\mbox{feature rich}} \Jabber{}/XMPP server.

\ejabberd{} is suitable for small servers, whether they need to be scalable or not, as well as extremely big servers.

%\subsection{Layout with example deployment (title needs a better name)}

%In this section there will be a graphical overview like these:\\
%\verb|http://www.tipic.com/var/timp/timp_dep.gif| \\
%\verb|http://www.jabber.com/images/jabber_Com_Platform.jpg| \\
%\verb|http://www.antepo.com/files/OPN45systemdatasheet.pdf| \\

%Some small images of Jabber clients that are known to work greatly with ejabberd. Less text!!!

%\subsection{Try It Today}

%(Not sure if I will include/finish this section for the next version.)

%\begin{itemize}
%\item Erlang REPOS
%\item Packages in distributions
%\item Windows binary
%\item source tar.gz
%\item Migration from Jabberd14 (and so also Jabberd2 because you can migrate from version 2 back to 14) and Jabber Inc. XCP possible.
%\end{itemize}

\newpage
\subsection{Key Features}
\label{sec:keyfeatures}
\ind{features!key features}

\quoting{Erlang seems to be tailor-made for writing stable, robust servers. --
Peter Saint-Andr\'e, Executive Director of the Jabber Software Foundation}

\ejabberd{} is:
\begin{itemize}
\item \marking{Multiplatform:} \ejabberd{} runs under Microsoft Windows and Unix derived systems such as Linux, FreeBSD and NetBSD.

\item \marking{Distributed:} You can run \ejabberd{} on a cluster of machines and all of them will serve the same \Jabber{} domain(s). When you need more capacity you can simply add a new cheap node to your cluster. Accordingly, you do not need to buy an expensive high-end machine to support tens of thousands concurrent users.

\item \marking{Fault-tolerant:} You can deploy an \ejabberd{} cluster so that all the information required for a properly working service will be replicated permanently on all nodes. This means that if one of the nodes crashes, the others will continue working without disruption. In addition, nodes also can be added or replaced ``on the fly''.

\item \marking{Administrator Friendly:} \ejabberd{} is built on top of the Open Source Erlang. As a result you do not need to install an external database, an external web server, amongst others because everything is already included, and ready to run out of the box. Other administrator benefits include:
\begin{itemize}
\item Comprehensive documentation.\moreinfo{ --- You can start in the \footahref{http://ejabberd.jabber.ru/book}{ejabberd Book}.}
\item Straightforward installers for Linux, Mac OS X, and Windows.\improved{}\moreinfo{ --- (\footahref{http://ejabberd.jabber.ru/screenshots-linux-installer}{Screenshots}).}
\item Web interface for administration tasks.\moreinfo{ --- With HTTPS secure access. \footahref{http://ejabberd.jabber.ru/online-demo-webadmin}{Demo}.}
\item Shared Roster Groups.\moreinfo{ --- The administrator can setup a common list of \Jabber{} users for all users on the server. Those users are virtually added to all rosters. They cannot be removed, but can be renamed or moved into different roster groups. Does not require client implementation. Not related to \jepref{0144} (Roster Item Exchange).\footahref{http://ejabberd.jabber.ru/screenshots-shared-roster-groups}{Screenshots})}
\item Command line administration tool.\improved{}\moreinfo{ --- Some basic administration tasks can be acomplished using the command line: register/remove users, backup/restore database, amongst others (\footahref{http://ejabberd.jabber.ru/screenshots-administration#ejabberdctl}{Screenshots}).}
\item Can integrate with existing authentication mechanisms.
\item Capability to send announce messages.\improved{}
\end{itemize}

\item \marking{Internationalized:} \ejabberd{} leads in internationalization. Hence it is very well suited in a globalized world. Related features are:
\begin{itemize}
\item Translated in 11 languages.\moreinfo{ --- More information is available \footahref{http://ejabberd.jabber.ru/localization}{here}.}
\item Support for \footahref{http://www.ietf.org/rfc/rfc3490.txt}{IDNA}.
\end{itemize}

\item \marking{Open Standards:} \ejabberd{} is the first Open Source Jabber server claiming to fully comply to the XMPP standard.
\begin{itemize}
\item Fully XMPP compliant \moreinfo{ --- ejabberd is fully compliant with XMPP Core 1.0 and XMPP IM 1.0. Check the \footahref{http://ejabberd.jabber.ru/protocols}{supported protocols}.}
\item XML-based protocol
\item \footahref{http://ejabberd.jabber.ru/protocols}{Many JEPs supported}.
\end{itemize}

\end{itemize}

\newpage

\subsection{Additional Features}
\label{sec:addfeatures}
\ind{features!additional features}

\quoting{ejabberd is making inroads to solving the "buggy incomplete server" problem --
Justin Karneges, Founder of the Psi and the Delta projects}

Besides common \Jabber{} server features, \ejabberd{} comes with a wide range of other features:
\begin{itemize}
\item Modular
\begin{itemize}
\item Load only the modules you want.
\item Extend \ejabberd{} with your own custom modules.\moreinfo{ --- A list of contributed modules and patches is available on the \footahref{http://ejabberd.jabber.ru/contributions}{contributions page}.}
\end{itemize}
\item Security
\begin{itemize}
\item SASL and STARTTLS for c2s and s2s connections.\improved{}
\item STARTTLS and Dialback s2s connections.
\item Web interface accessible via HTTPS secure access.
\end{itemize}
\item Databases
\begin{itemize}
\item Native MySQL support.\new{}
\item Native PostgreSQL support.
\item Mnesia.
\item ODBC data storage support. \moreinfo{ --- ODBC requests can be load
  balanced between several connections.}
\item Microsoft SQL Server support (via ODBC).\new{}
\end{itemize}
\item Authentication
\begin{itemize}
\item LDAP and ODBC. \moreinfo{ --- Accounts can authenticate in a LDAP server.}
\item External Authentication script.
\item Internal Authentication.
\end{itemize}
\item Others
\begin{itemize}
\item Compressing XML streams with Stream Compression (\jepref{0138}).\new{}
\item Interface with networks such as AIM, ICQ and MSN.
\item Statistics via Statistics Gathering (\jepref{0039}).
\item IPv6 support both for c2s and s2s connections.
\item \tjepref{0045}{Multi-User Chat} module with logging.\improved{}
\item Users Directory based on users vCards.
\item \tjepref{0060}{Publish-Subscribe} component.
\item Support for virtual hosting. \moreinfo{ --- Several \Jabber{} hosts can be hosted on the same \ejabberd{} instance. As simple as adding a new domain name to the list of hosts in the configuration file.}
\item \tjepref{0025}{HTTP Polling} service.
\item IRC transport.\improved{}
\end{itemize}
\end{itemize}

\section{How it Works}
\label{sec:howitworks}


A \Jabber{} domain is served by one or more \ejabberd{} nodes.  These nodes can
be run on different machines that are connected via a network.  They all must
have the ability to connect to port 4369 of all another nodes, and must have
the same magic cookie (see Erlang/OTP documentation, in other words the file
\texttt{\~{}ejabberd/.erlang.cookie} must be the same on all nodes). This is
needed because all nodes exchange information about connected users, S2S
connections, registered services, etc\ldots



Each \ejabberd{} node have following modules:
\begin{itemize}
\item router;
\item local router.
\item session manager;
\item S2S manager;
\end{itemize}


\subsection{Router}

This module is the main router of \Jabber{} packets on each node.  It routes
them based on their destinations domains.  It has two tables: local and global
routes.  First, domain of packet destination searched in local table, and if it
found, then the packet is routed to appropriate process.  If no, then it
searches in global table, and is routed to the appropriate \ejabberd{} node or
process.  If it does not exists in either tables, then it sent to the S2S
manager.


\subsection{Local Router}

This module routes packets which have a destination domain equal to this server
name.  If destination JID has a non-empty user part, then it routed to the
session manager, else it is processed depending on it's content.


\subsection{Session Manager}

This module routes packets to local users.  It searches for what user resource
packet must be sended via presence table.  If this resource is connected to
this node, it is routed to C2S process, if it connected via another node, then
the packet is sent to session manager on that node.


\subsection{S2S Manager}

This module routes packets to other \Jabber{} servers.  First, it checks if an
open S2S connection from the domain of the packet source to the domain of
packet destination already exists. If it is open on another node, then it
routes the packet to S2S manager on that node, if it is open on this node, then
it is routed to the process that serves this connection, and if a connection
does not exist, then it is opened and registered.




\section{XML Representation}
\label{sec:xmlrepr}

Each XML stanza is represented as the following tuple:
\begin{verbatim}
XMLElement = {xmlelement, Name, Attrs, [ElementOrCDATA]}
        Name = string()
        Attrs = [Attr]
        Attr = {Key, Val}
        Key = string()
        Val = string()
        ElementOrCDATA = XMLElement | CDATA
        CDATA = {xmlcdata, string()}
\end{verbatim}
E.\,g. this stanza:
\begin{verbatim}
<message to='test@conference.example.org' type='groupchat'>
  <body>test</body>
</message>
\end{verbatim}
is represented as the following structure:
\begin{verbatim}
{xmlelement, "message",
    [{"to", "test@conference.example.org"},
     {"type", "groupchat"}],
    [{xmlelement, "body",
         [],
         [{xmlcdata, "test"}]}]}}
\end{verbatim}



\section{Module \texttt{xml}}
\label{sec:xmlmod}

\begin{description}
\item{\verb|element_to_string(El) -> string()|}
\begin{verbatim}
El = XMLElement
\end{verbatim}
  Returns string representation of XML stanza \texttt{El}.

\item{\verb|crypt(S) -> string()|}
\begin{verbatim}
S = string()
\end{verbatim}
  Returns string which correspond to \texttt{S} with encoded XML special
  characters.

\item{\verb|remove_cdata(ECList) -> EList|}
\begin{verbatim}
ECList = [ElementOrCDATA]
EList = [XMLElement]
\end{verbatim}
  \texttt{EList} is a list of all non-CDATA elements of ECList.



\item{\verb|get_path_s(El, Path) -> Res|}
\begin{verbatim}
El = XMLElement
Path = [PathItem]
PathItem = PathElem | PathAttr | PathCDATA
PathElem = {elem, Name}
PathAttr = {attr, Name}
PathCDATA = cdata
Name = string()
Res = string() | XMLElement
\end{verbatim}
  If \texttt{Path} is empty, then returns \texttt{El}.  Else sequentially
  consider elements of \texttt{Path}.  Each element is one of:
  \begin{description}
  \item{\verb|{elem, Name}|} \texttt{Name} is name of subelement of
    \texttt{El}, if such element exists, then this element considered in
    following steps, else returns empty string.
  \item{\verb|{attr, Name}|} If \texttt{El} have attribute \texttt{Name}, then
    returns value of this attribute, else returns empty string.
  \item{\verb|cdata|} Returns CDATA of \texttt{El}.
  \end{description}

\item{TODO:}
\begin{verbatim}
         get_cdata/1, get_tag_cdata/1
         get_attr/2, get_attr_s/2
         get_tag_attr/2, get_tag_attr_s/2
         get_subtag/2
\end{verbatim}
\end{description}


\section{Module \texttt{xml\_stream}}
\label{sec:xmlstreammod}

\begin{description}
\item{\verb!parse_element(Str) -> XMLElement | {error, Err}!}
\begin{verbatim}
Str = string()
Err = term()
\end{verbatim}
  Parses \texttt{Str} using XML parser, returns either parsed element or error
  tuple.
\end{description}


\section{Modules}
\label{sec:emods}


%\subsection{gen\_mod behaviour}
%\label{sec:genmod}

%TBD

\subsection{Module gen\_iq\_handler}
\label{sec:geniqhandl}

The module \verb|gen_iq_handler| allows to easily write handlers for IQ packets
of particular XML namespaces that addressed to server or to users bare JIDs.

In this module the following functions are defined:
\begin{description}
\item{\verb|add_iq_handler(Component, Host, NS, Module, Function, Type)|}
\begin{verbatim}
Component = Module = Function = atom()
Host = NS = string()
Type = no_queue | one_queue | parallel
\end{verbatim}
  Registers function \verb|Module:Function| as handler for IQ packets on
  virtual host \verb|Host| that contain child of namespace \verb|NS| in
  \verb|Component|.  Queueing discipline is \verb|Type|.  There are at least
  two components defined:
  \begin{description}
  \item{\verb|ejabberd_local|} Handles packets that addressed to server JID;
  \item{\verb|ejabberd_sm|} Handles packets that addressed to users bare JIDs.
  \end{description}
\item{\verb|remove_iq_handler(Component, Host, NS)|}
\begin{verbatim}
Component = atom()
Host = NS = string()
\end{verbatim}
  Removes IQ handler on virtual host \verb|Host| for namespace \verb|NS| from
  \verb|Component|.
\end{description}

Handler function must have the following type:
\begin{description}
\item{\verb|Module:Function(From, To, IQ)|}
\begin{verbatim}
From = To = jid()
\end{verbatim}
\end{description}



\begin{verbatim}
-module(mod_cputime).

-behaviour(gen_mod).

-export([start/2,
         stop/1,
         process_local_iq/3]).

-include("ejabberd.hrl").
-include("jlib.hrl").

-define(NS_CPUTIME, "ejabberd:cputime").

start(Host, Opts) ->
    IQDisc = gen_mod:get_opt(iqdisc, Opts, one_queue),
    gen_iq_handler:add_iq_handler(ejabberd_local, Host, ?NS_CPUTIME,
                                  ?MODULE, process_local_iq, IQDisc).

stop(Host) ->
    gen_iq_handler:remove_iq_handler(ejabberd_local, Host, ?NS_CPUTIME).

process_local_iq(From, To, {iq, ID, Type, XMLNS, SubEl}) ->
    case Type of
        set ->
            {iq, ID, error, XMLNS,
             [SubEl, ?ERR_NOT_ALLOWED]};
        get ->
            CPUTime = element(1, erlang:statistics(runtime))/1000,
            SCPUTime = lists:flatten(io_lib:format("~.3f", CPUTime)),
            {iq, ID, result, XMLNS,
             [{xmlelement, "query",
               [{"xmlns", ?NS_CPUTIME}],
               [{xmlelement, "cputime", [], [{xmlcdata, SCPUTime}]}]}]}
    end.
\end{verbatim}


\subsection{Services}
\label{sec:services}

%TBD


%TODO: use \verb|proc_lib|
\begin{verbatim}
-module(mod_echo).

-behaviour(gen_mod).

-export([start/2, init/1, stop/1]).

-include("ejabberd.hrl").
-include("jlib.hrl").

start(Host, Opts) ->
    MyHost = gen_mod:get_opt(host, Opts, "echo." ++ Host),
    register(gen_mod:get_module_proc(Host, ?PROCNAME),
             spawn(?MODULE, init, [MyHost])).

init(Host) ->
    ejabberd_router:register_local_route(Host),
    loop(Host).

loop(Host) ->
    receive
        {route, From, To, Packet} ->
            ejabberd_router:route(To, From, Packet),
            loop(Host);
        stop ->
            ejabberd_router:unregister_route(Host),
            ok;
        _ ->
            loop(Host)
    end.

stop(Host) ->
    Proc = gen_mod:get_module_proc(Host, ?PROCNAME),
    Proc ! stop,
    {wait, Proc}.
\end{verbatim}



\end{document}
